\documentclass[a4paper,12pt]{scrartcl}    %attenzione! uso scratctl che permette sottotitolo
\usepackage{geometry}
 \geometry{
 a4paper,
 total={185mm,277mm},
 left=5mm,
 right=5mm,
 top=5mm,
 bottom=5mm,
 }


%Per le Figure
\usepackage[english]{babel}
\usepackage{graphicx}

\usepackage{amssymb}
\usepackage[leqno]{amsmath}
\usepackage{amsfonts}

\usepackage{tikz}
\usetikzlibrary{matrix}
\usepackage{dsfont}
\usepackage{graphicx}
\usepackage{hyperref}
\usepackage{enumerate}
\usepackage{cancel}
\usepackage{fourier}
\usepackage{mathtools}

\usepackage{pifont} %tick checkmark
\usepackage[symbol]{footmisc}
\renewcommand*{\thefootnote}{\fnsymbol{footnote}}

\usepackage{arydshln}

\usepackage{pbox}

%Common symbols
\usepackage[all]{../src/toninus-math-symbols}
%primo passo: creare un environment box personalizzato
%   Preambolo personalizzato 
%		Inscatolamento di teoremi, dimostrazioni e definizioni con diversi colori
%
%

%%%%%%%%%%%%%%%%%%%%%%%%%%%%%%%%%%%%%%%%%%%%%%%%%%%%%%%%%
% i colori di questa pagina http://en.wikibooks.org/wiki/LaTeX/Colors#The_68_standard_colors_known_to_dvips 
%\usepackage[usenames,dvipsnames]{color}


%%%%%%%%%%%%%%%%%%%%%%%%%%%%%%%%%%%%%%%%%%%%%%%%%%%%%%%%%
%con opzione per il tratteggio
	\usepackage[framemethod=TikZ]{mdframed}  
	\mdfdefinestyle{dotted}{ tikzsetting={ 
		draw= white, 
		dash pattern = on 3pt off 3pt 
	}	}
   



 	\usepackage{amssymb} 
   

 

%%%%%%%%%%%%%%%%%%%%%%%%%%%%%%%%%%%%%%%%%%%%%%%%%%%%%%%%%
%definisco l'ambiente theorem
\mdfdefinestyle{theoremstyle}{
	linecolor=blue,
	middlelinewidth=2pt,
	skipabove=10pt,
%	leftmargin=-10pt,
%	rightmargin=-10pt,
	everyline = true,
}

	\newmdtheoremenv[style=theoremstyle]{lemma}{Lemma}[section]
	\newmdtheoremenv[style=theoremstyle]{theorem}{Theorem}[section]
	\newmdtheoremenv[style=theoremstyle]{corollary}{Corollary}[section]
	\newmdtheoremenv[style=theoremstyle]{proposition}{Proposition}[section]


%   Fancy Version

\newcounter{Lemma}[section]
\newenvironment{Lemma}[1][]{%
  \stepcounter{Lemma}%
  \ifstrempty{#1}%
  {\mdfsetup{%
    frametitle={%
      \tikz[baseline=(current bounding box.east),outer sep=0pt]
      \node[line width=1pt,anchor=east,rectangle,draw=blue!20,fill=white]
    {\strut \color{blue}{Lemma}~\theLemma};}}
  }%
  {\mdfsetup{%
    frametitle={%
      \tikz[baseline=(current bounding box.east),outer sep=0pt]
      \node[line width=1pt,anchor=east,rectangle,draw=blue!20,fill=white]
    {\strut \color{blue}{Lemma}~\theLemma:~\color{gray}{#1}};}}%
  }%
  \mdfsetup{innertopmargin=10pt,linecolor=blue!20,%
            linewidth=1pt,topline=true,%
            frametitleaboveskip=\dimexpr-\ht\strutbox\relax,}
  \begin{mdframed}[]\relax%
  }{\end{mdframed}}



%%%%%%%%%%%%%%%%%%%%%%%%%%%%%%%%%%%%%%%%%%%%%%%%%%%%%%%%%
%Ipotesi con barra a sinistra

	\usetikzlibrary{calc}
%	\usepackage{fourier-orns}
	\tikzset{
  		box/.style={
      		rectangle,
      		draw=red,
      		fill=white,
      		scale=1,
      		overlay
    }		}

	\mdfdefinestyle{hypothesis}{%
 		hidealllines=true,leftline=true,
 		skipabove=12,skipbelow=12pt,
 		innertopmargin=0.4em,%
 		innerbottommargin=0.4em,%
 		innerrightmargin=0.7em,%
 		rightmargin=0.7em,%
 		innerleftmargin=1.7em,%
 		leftmargin=0.7em,%
 		middlelinewidth=.2em,%
 		linecolor=blue,%
 		firstextra={\path let \p1=(P), \p2=(O) in ($(\x2,0)+(0,\y1)$) 
                           node[box] {\textcolor{blue}{Hp:}};},%
 		secondextra={\path let \p1=(P), \p2=(O) in ($(\x2,0)+(0,\y1)$) 
                           node[box] {\textcolor{blue}{Hp:}};},%
 		middleextra={\path let \p1=(P), \p2=(O) in ($(\x2,0)+(0,\y1)$) 
                           node[box] {\textcolor{blue}{Hp:}};},%
 		singleextra={\path let \p1=(P), \p2=(O) in ($(\x2,0)+(0,\y1)$) 
                           node[box] {\textcolor{blue}{Hp:}};},%
	}

\newmdenv[style=hypothesis]{hypothesis}




%%%%%%%%%%%%%%%%%%%%%%%%%%%%%%%%%%%%%%%%%%%%%%%%%%%%%%%%%
%Tesi con barra a sinistra

	\mdfdefinestyle{thesis}{%
 		hidealllines=true,leftline=true,
 		skipabove=12,skipbelow=12pt,
 		innertopmargin=0.4em,%
 		innerbottommargin=0.4em,%
 		innerrightmargin=0.7em,%
 		rightmargin=0.7em,%
 		innerleftmargin=1.7em,%
 		leftmargin=0.7em,%
 		middlelinewidth=.2em,%
 		linecolor=blue,%
 		firstextra={\path let \p1=(P), \p2=(O) in ($(\x2,0)+(0,\y1)$) 
                           node[box] {\textcolor{blue}{Th:}};},%
 		secondextra={\path let \p1=(P), \p2=(O) in ($(\x2,0)+(0,\y1)$) 
                           node[box] {\textcolor{blue}{Th:}};},%
 		middleextra={\path let \p1=(P), \p2=(O) in ($(\x2,0)+(0,\y1)$) 
                           node[box] {\textcolor{blue}{Th:}};},%
 		singleextra={\path let \p1=(P), \p2=(O) in ($(\x2,0)+(0,\y1)$) 
                           node[box] {\textcolor{blue}{Th:}};},%
	}

\newmdenv[style=thesis]{thesis}


%%%%%%%%%%%%%%%%%%%%%%%%%%%%%%%%%%%%%%%%%%%%%%%%%%%%%%%%%
%Dimostrazione, Tratteggiato da appendere sotto la proposizione
\mdfdefinestyle{proofstyle}{
	style = dotted,
	linecolor=blue,
	skipabove=0pt,
	topline =false,
}
 
 
    \newenvironment{proof}%
    {%
    \begin{mdframed}[style = proofstyle,frametitle={Proof:}]
    }
    {%
    	\begin{flushright}
			$\square$
		\end{flushright}
    \end{mdframed}%
    }%

%%%%%%%%%%%%%%%%%%%%%%%%%%%%%%%%%%%%%%%%%%%%%%%%%%%%%%%%%
%DEFINIZIONE
\mdfdefinestyle{definitionstyle}{
	linecolor=red,
	middlelinewidth=2pt,
	frametitlerule=true,
	frametitlerulecolor=green!60,
%	frametitlerulewidth =1pt,
%	innertopmargin= \topskip ,
	skipabove=3pt,
}

\mdtheorem[ style = definitionstyle ]{definition}{Definition}

%%%%%%%%%%%%%%%%%%%%%%%%%%%%%%%%%%%%%%%%%%%%%%%%%%%%%%%%%
%NOTAZIONE   (WIP)
\mdfdefinestyle{notationstyle}{
	roundcorner=5pt,
	linecolor=green,
	middlelinewidth=2pt,
	frametitlerule=true,
	frametitlerulecolor=green!60,
%	frametitlerulewidth =1pt,
%	innertopmargin= \topskip ,
	frametitle={Notation fixing},
	skipabove=3pt,
}

\newmdenv[style=notationstyle]{notationfix}





%%%%%%%%%%%%%%%%%%%%%%%%%%%%%%%%%%%%%%%%%%%%%%%%%%%%%%%%%
%TAKE AWAY MASSAGE (WIP)
\mdfdefinestyle{takeawaystyle}{
	linecolor=gray,
	middlelinewidth=1pt,
	frametitlerule=true,
	frametitle={Take Away Message},
	skipabove=3pt,
}

\newmdenv[style=takeawaystyle]{TAM}





%%%%%%%%%%%%%%%%%%%%%%%%%%%%%%%%%%%%%%%%%%%%%%%%%%%%%%%%%
% EXAMPLE  (WIP)
\mdfdefinestyle{examplestyle}{
  	linecolor=yellow,
   % frametitle={\textbf{Example:}\colorbox{white}{\space#1\space}},
    innertopmargin=10pt,
    frametitleaboveskip=-\ht\strutbox,
    frametitlealignment=\center
}

\mdtheorem[style=examplestyle]{example}{Example: }

% Fancy http://tex.stackexchange.com/questions/69148/how-to-insert-title-in-mdframed





%%%%%%%%%%%%%%%%%%%%%%%%%%%%%%%%%%%%%%%%%%%%%%%%%%%%%%%%%
% Observation (WIP)
\mdfdefinestyle{observationstyle}{
	linecolor=magenta,
	middlelinewidth=1pt,
	frametitlerule=true,
%	frametitle={Observation},	
	skipabove=3pt,
}

%\newmdenv[style=observationstyle]{observation}
\mdtheorem[ style = observationstyle ]{observation}{Observation}




%%%%%%%%%%%%%%%%%%%%%%%%%%%%%%%%%%%%%%%%%%%%%%%%%%%%%%%%%
% Remark (WIP)
\mdfdefinestyle{remarkstyle}{
	linecolor=cyan,
	middlelinewidth=1pt,
	frametitlerule=true,
	frametitle={Remark:},	
	skipabove=3pt,
}

\newmdenv[style=remarkstyle]{remark}

%%%%%%%%%%%%%%%%%%%%%%%%%%%%%%%%%%%%%%%%%%%%%%%%%%%%%%%%%
% NotaBene (WIP)
\newenvironment{NB}[1]
    {
		\begin{description}
			\item[N.B. :]
			
			
    }
    { 
		\end{description}    
	}

%%%%%%%%%%%%%%%%%%%%%%%%%%%%%%%%%%%%%%%%%%%%%%%%%%%%%%%%%
% Danger box (WIP).
% It encapsulates a piece of text that needs revision
% http://tex.stackexchange.com/questions/52023/mdframed-put-something-on-the-start-of-one-vertical-left-rule

\definecolor{warningColor}{named}{red}
\tikzset{
  warningsymbol/.style={
      rectangle,
      draw=warningColor,
      fill=white,
      scale=1,
      overlay}
}

\mdfdefinestyle{warning}{%
 hidealllines=true,leftline=true,
 skipabove=12,skipbelow=12pt,
 innertopmargin=0.4em,%
 innerbottommargin=0.4em,%
 innerrightmargin=0.7em,%
 rightmargin=0.7em,%
 innerleftmargin=1.7em,%
 leftmargin=0.7em,%
 middlelinewidth=.2em,%
 linecolor=warningColor,%
 fontcolor=warningColor,%
 firstextra={\path let \p1=(P), \p2=(O) in ($(\x2,0)+0.5*(0,\y1)$) 
                           node[warningsymbol] {\danger};},%
 secondextra={\path let \p1=(P), \p2=(O) in ($(\x2,0)+0.5*(0,\y1)$) 
                           node[warningsymbol] {\danger};},%
 middleextra={\path let \p1=(P), \p2=(O) in ($(\x2,0)+0.5*(0,\y1)$) 
                           node[warningsymbol] {\danger};},%
 singleextra={\path let \p1=(P), \p2=(O) in ($(\x2,0)+0.5*(0,\y1)$) 
                           node[warningsymbol] {\danger};},%
}

\newmdenv[style=warning]{Warning}


\usepackage{colortbl}%http://tex.stackexchange.com/questions/50349/color-only-a-cell-of-a-table
\usepackage{multirow}

\newcommand{\OpA}{\otimes}
\newcommand{\OpB}{\oplus}

\newcommand{\needed}{\cellcolor{green!25} needed}
\newcommand{\unneeded}{ \cellcolor{red!25} unneeded}
\newcommand{\xneeded}{\cellcolor{green!25} \ding{51}}
\newcommand{\xunneeded}{ \cellcolor{red!25} \ding{55}}


\usepackage{pdflscape}


\begin{document}

\begin{landscape}
    \thispagestyle{empty}
    \noindent


%+------------------------------------------------------------------------------------+
% |		Wip																			|
%+------------------------------------------------------------------------------------+

%  Titolo
	\title{Prolegomena on (some) Abstact Algebra}
	\subtitle{Elementary Algebraic structures zoology}
	\author{Tony}
	\date{\today}
\maketitle

	\begin{minipage}[c][\textheight]{0.30 \linewidth}
	    \section{(WIP)Work in Progress}
	    \mbox{}\\
		\begin{itemize}
			\item Fin qui, strutture algebriche dal punto di vista dell'algebra elementare. Mancano:
				\begin{itemize}
					\item Relazioni (c'è dualismo tra le relazioni e le mappe binarie: le seconde sono particolari relazioni mentre le primi formano strutture. eg: i reticoli formano un semigruppo
					\item approccio diagrammatico alle relazioni.
					\item punto di vista della \emph{universal algebra}
				\end{itemize}
			\item Gli spazi vettoriali vanno messi vicino ai moduli
			\item le algebre sono su anello e su campo
			\item Gli anelli sono un gruppo abeliano su cui agisce se stesso in modo non abeliano
			\item i moduli sono un gruppo abeliano su cui agisce un anello
			\item le algebre sono un p su cui agisce q
			\item 
		\end{itemize}	
	\end{minipage}
	%
	\hspace{1cm}
	%
	\begin{minipage}[t][]{0.60 \linewidth}
		%
		\begin{minipage}[c]{0,5\textwidth}	
			\begin{tabular}{|l|p{2cm}|} %Magma
			  \hline
			  \multicolumn{2}{c}{\cellcolor{yellow!25}Magma} \\
			  \hline
			   \cellcolor{blue!25} Closure&  \\
			    \cellcolor{red!25} Associativity&  \\
			    \cellcolor{red!25} Identity&  \\
			    \cellcolor{red!25} Inverse&  \\
			    \cellcolor{red!25} Commutativity&  \\
			  \hline
			\end{tabular}
			\vfill
			\begin{tabular}{|l|p{2cm}|} %Monoid
			  \hline
			  \multicolumn{2}{c}{\cellcolor{yellow!25}Monoid} \\
			  \hline
			   \cellcolor{blue!25} Closure&  \\
			    \cellcolor{blue!25} Associativity&  \\
			    \cellcolor{blue!25} Identity&  \\
			    \cellcolor{red!25} Inverse&  \\
			    \cellcolor{red!25} Commutativity&  \\
			  \hline
			\end{tabular}
			\vfill
			\begin{tabular}{|l|p{2cm}|} %Group
			  \hline
			  \multicolumn{2}{c}{\cellcolor{yellow!25}Group} \\
			  \hline
			   \cellcolor{blue!25} Closure&  \\
			    \cellcolor{blue!25} Associativity&  \\
			    \cellcolor{blue!25} Identity&  \\
			    \cellcolor{blue!25} Inverse&  \\
			    \cellcolor{red!25} Commutativity&  \\
			  \hline
			\end{tabular}
		\end{minipage}
		%
		\hspace{1cm}
		%
		\begin{minipage}[c]{0,5\textwidth}	
\begin{tabular}{|l|p{2cm}|} %SemiGroup
			  \hline
			  \multicolumn{2}{c}{\cellcolor{yellow!25}SemiGroup} \\
			  \hline
			   \cellcolor{blue!25} Closure&  \\
			    \cellcolor{blue!25} Associativity&  \\
			    \cellcolor{red!25} Identity&  \\
			    \cellcolor{red!25} Inverse&  \\
			    \cellcolor{red!25} Commutativity&  \\
			  \hline
			\end{tabular}
			\vfill
			\begin{tabular}{|l|p{2cm}|} %Commutative Monoid
			  \hline
			  \multicolumn{2}{c}{\cellcolor{yellow!25}Commutative Monoid} \\
			  \hline
			   \cellcolor{blue!25} Closure&  \\
			    \cellcolor{blue!25} Associativity&  \\
			    \cellcolor{blue!25} Identity&  \\
			    \cellcolor{red!25} Inverse&  \\
			    \cellcolor{blue!25} Commutativity&  \\
			  \hline
			\end{tabular}
			\vfill
			\begin{tabular}{|l|p{2cm}|} %Abelian Group
			  \hline
			  \multicolumn{2}{c}{\cellcolor{yellow!25}Abelian Group} \\
			  \hline
			   \cellcolor{blue!25} Closure&  \\
			    \cellcolor{blue!25} Associativity&  \\
			    \cellcolor{blue!25} Identity&  \\
			    \cellcolor{blue!25} Inverse&  \\
			    \cellcolor{blue!25} Commutativity&  \\
			  \hline
			\end{tabular}
			\vfill
		\end{minipage}
		%
	\end{minipage}	






%-_-_-_-_-_-_-_-_-_-_-_-_-_-_-_-_-_-_-_-_-_-_-_-_-_-_-_-_-_-_-_-_-_-_-_-_-_-_-_-_-_-_-_-_-_-_-_-_-_-_-_-_-_-_-_
\newpage

%+------------------------------------------------------------------------------------+
% |		Relations																			|
%+------------------------------------------------------------------------------------+
\newpage
	\begin{minipage}[t][]{0.30 \linewidth}
	    \section*{Relations}
	    \mbox{}\\
		Consider a set $X$, a relation is a subset $\rho \subset X \times X$.
		Consider $x,y \in X$ we say $x \rho y $ iff $(x,y)\in \rho$.
		
		Basics:
		
		
		Diagram:
			
	\end{minipage}
	%
	\hspace{1cm}
	%
	\begin{minipage}[t][]{0.60 \linewidth}
    	\begin{center}
    	Depending on the properties satisfied by $\rho$, the subset is called:
    	\begin{tabular}{|c|c|c|c|c|c|c|c|c|c|c|c|}
    		\hline
    		& \rotatebox[origin=c]{90}{Relation} & \rotatebox[origin=c]{90}{Symmetric} & \rotatebox[origin=c]{90}{Antisymmetric} 
    		& \rotatebox[origin=c]{90}{Transitive} & \rotatebox[origin=c]{90}{Reflexive} & \rotatebox[origin=c]{90}{Total ordered} 
    		& \rotatebox[origin=c]{90}{Minimal Condition} & \rotatebox[origin=c]{90}{With lower bounds} & \rotatebox[origin=c]{90}{Complete} 
    		& \rotatebox[origin=c]{90}{Univocal} & \rotatebox[origin=c]{90}{Everywhere defined} \\
    		\hline
    		Binary Relation & \xneeded & \xunneeded & \xunneeded & \xunneeded & \xunneeded & \xunneeded & \xunneeded & \xunneeded & \xunneeded & \xunneeded & \xunneeded  \\
    		Equivalence & \xneeded & \xneeded & \xunneeded & \xneeded & \xneeded & \xunneeded & \xunneeded & \xunneeded & \xunneeded & \xunneeded & \xunneeded  \\
    		Poset & \xneeded & \xunneeded & \xneeded & \xneeded & \xneeded & \xunneeded & \xunneeded & \xunneeded & \xunneeded & \xunneeded & \xunneeded  \\
    		Total Order & \xneeded & \xunneeded & \xneeded & \xneeded & \xneeded & \xneeded & \xunneeded & \xunneeded & \xunneeded & \xunneeded & \xunneeded  \\
    		Lower Semilattice & \xneeded & \xunneeded & \xneeded & \xneeded & \xneeded & \xneeded & \xneeded & \xneeded & \xunneeded & \xunneeded & \xunneeded  \\
    		Complete Semilattice & \xneeded & \xunneeded & \xneeded & \xneeded & \xneeded & \xneeded & \xneeded & \xneeded & \xneeded & \xunneeded & \xunneeded  \\
    		Partial Map & \xneeded & \xunneeded & \xunneeded & \xunneeded & \xunneeded & \xunneeded & \xunneeded & \xunneeded & \xunneeded & \xneeded & \xunneeded  \\
    		Function & \xneeded & \xunneeded & \xunneeded & \xunneeded & \xunneeded & \xunneeded & \xunneeded & \xunneeded & \xunneeded & \xneeded & \xneeded  \\
    		\hline
    	\end{tabular}
    	\end{center}
    	\begin{minipage}[t][]{0.45 \linewidth}
		%
		\fbox{
		  \parbox{\textwidth}{
				\emph{Relation} 
				\begin{equation}
					\rho \in \mathcal{P}(X \times X ) \label{Relation}
				\end{equation}
		  }
		}
		\vfill
		%
		\fbox{
		  \parbox{\textwidth}{
				\emph{Symmetric} 
				\begin{equation}
					x \rho y \Rightarrow y \rho x  \label{Symmetric}
				\end{equation}
		  }
		}
		\vfill
		%
		\fbox{
		  \parbox{\textwidth}{
				\emph{AntiSymmetric} 
				\begin{equation}
					x \rho y \; \wedge y \rho x \Rightarrow x=y  \label{Antisymmetric}
				\end{equation}
		  }
		}
		\vfill
		%
		\fbox{
		  \parbox{\textwidth}{
				\emph{Transitive} 
				\begin{equation}
					x \rho y \; \wedge y \rho z \Rightarrow x \rho z  \label{Transitive}
				\end{equation}
		  }
		}
		\vfill
		%
		\fbox{
		  \parbox{\textwidth}{
				\emph{Reflexive} 
				\begin{equation}
					\forall x \in X \quad x \rho x \label{Reflexive}
				\end{equation}
		  }
		}
		\vfill
		\end{minipage}
		%
		\hspace{1cm}
		%		
		\begin{minipage}[t][]{0.450 \linewidth}
		\fbox{
		  \parbox{\textwidth}{
				\emph{Total Ordered} 
				\begin{equation}
					\forall x,y \in X \quad x \rho y \vee y \rho x \label{Totalorder}
				\end{equation}
		  }
		}
		\vfill
		%
		\fbox{
		  \parbox{\textwidth}{
				\emph{Minimal Condition} 
				\begin{equation}
					\forall Y \subset \rho (\neq \emptyset)  \exists a_I \textrm{minimal element} \label{MinimalCond}
				\end{equation}
				\begin{displaymath}
					(\forall y \in Y) y \leq a_I \Rightarrow y = a_I
				\end{displaymath}
		  }
		}
		\vfill
		%
		\fbox{
		  \parbox{\textwidth}{
				\emph{Exists lower bound} 
				\begin{equation}
					\exists x\wedge y \forall x,y\in X\label{Lowerbound}
				\end{equation}
		  }
		}
		\vfill
		%
		\fbox{
		  \parbox{\textwidth}{
				\emph{Complete lower bound} 
				\begin{equation}
					\exists \bigwedge\{y : y \in Y\} \quad \forall Y \subset X \label{completeness}
				\end{equation}
		  }
		}		
		\vfill
		\end{minipage}	
	\end{minipage}
	
	\end{landscape}
\end{document}